The CMS RPC upgrade for Phase-2~\cite{muon_tdr} comprehends \textbf{(1) the replacement of a the current Link System}, which connects the Front-End Boards (FEBs) to the trigger processors, by a new one, redesigned one from scratch and \textbf{(2) the extension of the pseudorapidity coverage of the RPC system}, by adding new chambers from $|\eta| = 1.9$ up to 2.4. Those new chamber will be assembled with a Improved Resistive Plate Chambers (iRPC) technology, which does the readout of signals in both ends of the strip, allowing a 2-dimensional hit reconstruction. 

% The timing difference per hit and strip, is used by the iRPC Front-End electronics to estimate the spatial position of the hit in the longitudinal direction. The current RPC chambers can only read the transverse hit position.

\begin{wrapfigure}[16]{r}{0.65\textwidth}
    \caption{\footnotesize CMS Muon system for the Phase-2 Upgrade.}
    \includegraphics[width=0.65\textwidth]{uioposter-images/cms_muon}
    \label{cms_muon_upgrade}
\end{wrapfigure}


These upgrades would, in the expected high rate of the HL-LHC scenario:
\begin{itemize}
    \item enhance the redundancy of the CMS Muon System;
    \item resolve ambiguities in the Endcap triggering;
    \item allow improvements of the RPC system to Trigger and reconstruction. 
\end{itemize} 

%
% Both upgrades are important in order to cope with expected high rate of the HL-LHC scenario, in which a Inst. Luminosity of $5 \times 10^{34}$ $cm^{-2}s^{-1}$ would provide a background rate up to 700 $Hz/cm^2$ (for present chambers, already including a safety factor of 3). Also, the upgrades would enhance the redundancy of the CMS Muon System, resolve ambiguities in the Endcap triggering and allow improvements of the RPC system to Trigger and reconstruction. Figure 1 presents a quadrant of the CMS Muon system, showing Drift Tubes (DT) chambers in yellow, RPCs in light blue, and Cathode Strip Chambers (CSCs) in green. The locations of new forward muon detectors for the HL-LHC project are indicated in red for Gas Electron Multiplier (GEM) stations (ME0, GE1/1, and GE2/1) and violet for improved RPC stations (RE3/1 and RE4/1).



Figure 1 presents a quadrant of the CMS Muon system for the Phase-2 Upgrade. The locations of new forward muon detectors for the HL-LHC project are indicated in red for Gas Electron Multiplier (GEM) stations (ME0, GE1/1, and GE2/1) and violet for improved RPC stations (RE3/1 and RE4/1).

% Figure 1 presets a quadrant of the CMS Muon system for the Phase-2 Upgrade, showing Drift Tubes (DT) chambers in yellow, RPCs in light blue, and Cathode Strip Chambers (CSCs) in green. The locations of new forward muon detectors for the HL-LHC project are indicated in red for Gas Electron Multiplier (GEM) stations (ME0, GE1/1, and GE2/1) and violet for improved RPC stations (RE3/1 and RE4/1).